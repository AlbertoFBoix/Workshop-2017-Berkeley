\documentclass[twoside,12pt, leqno]{amsart}
\usepackage{amsmath,amscd,amsthm,amssymb,amsxtra,latexsym,epsfig,epic,graphics}
\usepackage[matrix,arrow,curve]{xy}
\usepackage{graphicx}
\usepackage{diagrams}
\usepackage{tikz,color}  %TikZ
%\usepackage{amsrefs}
%%%%%%%%%%%%%%%%%%%%%%%%%%%%%%%%%%%%%%%%%
%\textwidth16cm
%\textheig\codim20cm
%\topmargin-2cm
\oddsidemargin.8cm
\evensidemargin1cm

%%%%%Definitions
\input preamble.tex
\def\e{{\epsilon}}
\def\TU{{\bf U}}
\def\AA{{\mathbb A}}
\def\BB{{\mathbb B}}
\def\bB{{\mathbb B}}
\def\PP{{\mathbb P}}
\def\P{{\mathbb P}}
\def\QQ{{\mathbb Q}}
\def\FF{{\mathbb F}}
\def\facet{{\bf facet}}
\def\image{{\rm image}}
\def\cE{{\cal E}}
\def\cF{{\cal F}}
\def\cG{{\cal G}}
\def\cH{{\cal H}}
\def\cHom{{{\cal H}om}}
\def\fix#1{{\bf ***Fix:} #1 {\bf ***}}
\def\david#1{{\bf *** David:} #1 {\bf ***}}
\DeclareMathOperator{\rH}{{\rm H}}
\def\fC{{\mathfrak C}}
\def\Tr{{\rm Tr}}
\def\bC{{\mathbb C}}
\def\Gr{{\rm Gr}}
\def\CI{{\mathcal I}}
\def\CH{{\mathcal H}}
%\def\CCH{{\mathcal {CNT}}}
\def\CCH{{\mathcal {HC}}}
\def\rH{{\rm H}}

\def\soc{{\rm soc\,}}
\def\jacobian{{\rm Jac}}
\def\Rbar{{\overline R}}
\def\Ibar{{\overline I}}
\def\mm{{\frak m}}
\def\RR{{\mathcal R}}
\def\Trace{{\rm Tr}}

\def\CO{{\mathcal O}}
\def\CT{{\mathcal T}}
\def\CHom{{\mathcal Hom}}
\def\Spec{{{\rm Spec}\,}}
\def\cone{{{\rm cone}\,}}

\def\tR{{\tilde R}}
\def\tI{{\tilde I}}
\def\tJ{{\tilde J}}
\def\tK{{\tilde K}}
\def\tH{{\tilde H}}
\def\tF{{\tilde F}}

\newarrow{Iso} -----

\def\Abar{{\overline A}}
\def\Rbar{{\overline R}}
\def\Ibar{{\overline I}}
\def\Jbar{{\overline J}}
\def\Kbar{{\overline K}}
\def\abar{{\overline \alpha}}
\def\bbar{{\overline \beta}}
\def\m{{\frak m}}
\def\Rbar{{\overline R}}

\def\gr{{\rm gr}}

\def\lbracket{{[\kern-1.5pt[}}
\def\rbracket{{]\kern-1.5pt]}}

\def\seq#1#2{{#1_{1},\dots,#1_{#2}}}
\def\ff#1{{f_{1},\dots, f_{#1}}}

\makeatletter
\def\Ddots{\mathinner{\mkern1mu\raise\p@
\vbox{\kern7\p@\hbox{.}}\mkern2mu
\raise4\p@\hbox{.}\mkern2mu\raise7\p@\hbox{.}\mkern1mu}}
\makeatother


%%%%%%%%%%%%%%%%%%Silvio's macros for the diagrams
\usepackage{times}
\newdimen\x \x=12pt

%\usepackage{mat\codimime}
\usepackage{color}

%\usepackage{color}
%\usepackage[usenames,dvipsnames,svgnames,table]{xcolor}

\usepackage[breaklinks,bookmarksopen,bookmarksnumbered,urlcolor=blue]{hyperref}
\hypersetup{colorlinks=true,backref=true,citecolor=blue}

%\pagestyle{myheadings}
%\date{April 2013-December 2015}
\author{David Eisenbud}

\title{The Rees Algebra package in Macaulay2}
\begin{document}

\begin{abstract}
This note describes version 2.0 of the Macaulay2 package ReesAlgebra.m2.
\end{abstract}

\maketitle

\section*{Introduction}
A central construction in modern commutative algebra starts from
an ideal $I$ in a commutative ring $R$, and produces the \emph{Rees algebra}
$$
\RR(I) :=  R\oplus I\oplus I^2\oplus I^3\oplus\cdots \cong R[It]\subset R[t]
$$ 
where $R[t]$ denotes the polynomial algebra in one variable $t$ over $R$. For basics on Rees algebras, see \cite{Vasconcelos} and \cite{Huneke-****}.

From the point of view of algebraic geometry, the Rees algebra $\RR(I)$ is a homogeneous
coordinate ring for the graph of a rational map whose total space is the blowup
of $\Spec R$ along the scheme defined by $I$.
 (In fact, the  ``Rees Algebra''
     is sometimes called the ``blowup algebra''.)  
     

     Rees Algebras were first studied in the algebraic context by
      by David Rees, in the
     now famous paper~\cite{Rees}. Actually
     Rees mainly studied the ring 
     $R[It,t^{-1}]$, now also called the \emph{extended Rees
     Algebra} of I. 
     
There are several possible ways of extending the Rees algebra construction from ideals to modules. For simplicity we will henceforward only consider finitely generated modules
over Noetherian rings. Huneke and Ulrich and I argued in~\cite{EHU} that the most natural is
to think of $R[It]$ as the image of the map of symmetric algebras
$\Sym(\phi): \Sym_R(I) \to \Sym_R(R) = R[t]$, and to generalize it to the case of
an arbitrary finitely generated module $M$ by setting
$$
\RR(M) = \image \Sym(\phi)
$$ 
where $\phi$ is the \emph{versal} map from $M$ to a free module: If we denote
by $\phi_1,\dots, \phi_m$ a set of generators for $\Hom_R(M,R)$, then the versal
map is obtained by composing the direct sum 
$$
\oplus_{i=1}^m\phi_i
$$ 
with the diagonal 
embedding 
$$
M \to \oplus_{i=1}^mM.
$$
 It has the property that any map from $M$ to a free module factors (non-uniquely, in general) through $\phi$. Though this is not immediate, the Rees algebra of an ideal in any Noetherian ring, in this sense,
 is the same as the Rees algebra in the classical sense, and in most cases one can take
 any embedding of the module into a free module in the definition:
 
\begin{theorem}\label{good cases} [Eisenbud-Huneke-Ulrich, Thms 0.2 and 1.4] Let $R$ be a Noetherian ring
     and let $M$ be a finitely generated $R$-module. Let $\phi: M \to G$
     be the universal embedding of $M$ in a free module, and let
     $\psi: M \to G'$  be any inclusion of $M$ into a free module $G'$. 
     If $R$ is torsion-free over $\ZZ$
     or $R$ is unmixed and generically Gorenstein or $M$ is free locally
     at each associated prime of $R$, or $G=R$, then the image of $\Sym(\phi)$ and the
     image of $\Sym(\psi)$ are naturally isomorphic.
\end{theorem}

 The package ReesAlgebra.m2 was originally written around 2002 (by me and Mike Stillman, if memory serves) to implement this construction. Subsequently
Amelia Taylor, 
Sorin Popescu,
and, at the Macaulay2 Workgroup in July 2017, 
Ilir Dema,
Whitney Liske, and
Zhangchi Chen
all contributed.


The package now includes routines
for computing many of the invariants of an ideal or module
defined in terms of Rees algebras. This is in fact its primary utility, since Rees Algebras
of modules other than ideals are comparatively little studied. We now describe the central routines implemented in the package. It is a notoriously hard problem to describe Rees algebras by generators and relations in general cases, and the intuition provided by 
computations using this package has proven useful.

\section{The Rees Algebra and its relations:\\ {\tt reesIdeal, reesAlgebra, universalEmbedding, symmetricKernel, isLinearType,
jacobianDual,
associatedGradedRing, normalCone, multiplicity}}

The central routine, reesIdeal (with synonym reesAlgebraIdeal) computes an ideal
defining the Rees algebra $\RR(M)$ as a quotient of a polynomial ring over $R$ from a free presentation of $M$. From the Rees ideal we immediately get
${\tt reesAlgebra}\ M$. In the case when $M$ is an ideal in $R$ we also compute
the important ${\tt associatedGradedRing\ }M = \RR(M)/M$ (and the more geometric sounding but identical ${\tt normalCone\ }M$.)  
If $I$ is a (homogeneous) ideal primary to the
maximal ideal of a standard graded ring $R$ we compute the
Hilbert-Samuel multiplicity of $I$ with ${\tt multiplicity\ }I$.

We now describe the basic computation. Suppose that $M$ has 
a set of generators represented by a map from a free module,
a free presentation
$$
 F\rTo^\alpha M\to 0,
$$
and suppose  $F = R^n$. The symmetric algebra of $F$ over $R$ is then a polynomial ring
$\Sym_R(F_0) = R[t_1,\dots, t_n]$ on $n$ new indeterminates $t_1,\dots, t_n$. By the universal
property of the symmetric algebra there is a canonical surjection
$\Sym_R(F)\rightarrow \Sym_R(M)$, so we may compute the Rees algebra of $M$ as
a quotient of the  $\Sym_R(F)$. The call
$$
I = {\tt reesIdeal}\ M
$$
first calls ${\tt universalEmbedding\ }M$ to compute the versal map from $M$ to a free module $\beta: M\to G$. The call ${\tt symmetricKernel\ } \alpha\circ\beta$  then constructs the map of symmetric algebras $\beta\circ \alpha:\Sym_R(F)\to \Sym_R(G)$ and  calls
 the built-in Macaulay2 routine to compute the kernel 
$$
I = {\tt reesIdeal\ }M = \ker \Sym(\beta\circ\alpha): \Sym_R(F) \to \Sym_R(G).
$$

There is a different way of computing the Rees algebra that is often much more efficient. It begins by constructing the symmetric algebra of $M$, and uses the observation that the construction of the Rees algebra commutes with localization. See \cite[Appendix ****]{E} for the necessary facts about symmetric algebras.

Suppose that $M$ has a free presentation
$$
G\rTo^\alpha F\rTo^\epsilon M\to 0.
$$
The right exactness of the symmetric algebra functor implies that the symmetric algebra of $M$ is the quotient of $\Sym_R(F)$ by an ideal $I_0$ that is
generated by the
entries of the matrix
$$
\begin{pmatrix}
 t_1&\dots&t_n
 \end{pmatrix}
 \circ \phi
$$
(where we have identified $\phi$ with $\Sym_R(F)\otimes_R\phi$).
Thus $I_0$ is generated by polynomials that are linear in the variables $t_i$ (and because
$M$ is the degree 1 part of $\RR(M)$, these are the only linear forms in the $t_i$ in the
Rees ideal.)

If $f\in R$ is an element such that $M[f^{-1}]$ is free on generators $g_1,\dots, g_n$, it follows that after inverting $f$ the Rees algebra of $M$ becomes a polynomial ring over $R[f^{-1}]$ on
indeterminates corresponding to the $g_i$.
$$
\RR(M)[f^{-1}] = \Sym_R(M[f^{-1}]) = R[G_1,\dots, G_n]
$$

 Now suppose in addition that $f$ is a non-zerodivisor in $R$. In the diagram
$$
\begin{diagram}
 \Sym_R(F) &\rTo^\alpha &\Sym_R(M)&\rTo^\beta &\Sym_R(G)\\
 \dTo&&  \dTo&& \dTo\\
 \Sym_R(F)[f^{-1}] &\rTo^\alpha &\Sym_R(M)[f^{-1}]&\rTo^\beta &\Sym_R(G)[f^{-1}]\\
\end{diagram}
 $$
 the two outer vertical maps are inclusions, and it follows that the Rees ideal, which is the
 kernel of the map $\RR(F) = \Sym_R(F) \to \RR(M)$, is equal to the intersection
 of $\RR(F)$ with the kernel of
 $\Sym_R(F)[f^{-1}] \rTo^\beta \Sym_R(G)[f^{-1}]$. This intersection
 may be computed as $I_0:f^\infty$. The call
 $$
 {\tt reesIdeal}(I, f)
 $$
 computes the Rees ideal in this way.
 
More generally, we say that a module $N$ is {\em of linear type} if
the Rees ideal of $M$ is equal to the ideal of the symmetric algebra of $M$; 
for example, any complete intersection ideal is of linear type, and the condition
can be tested by the call
$$
{\tt isLinearType}\ M.
$$
The procedure above really requires only that $f$ be a non-zerodivisor in $R$ and
that $M[f^{-1}]$ be of linear type over $R[f^{-1}]$.

\section{Reductions and the special fiber: {\tt isReduction, minimalReduction, reductionNumber
specialFiber,
specialFiberIdeal,
analyticSpread}}

A \emph{reduction} $J$ of an ideal $I$ is an sub-ideal $J\subset I$ over which $I$ is
\emph{integrally dependent}. In concrete terms this means that there is some integer $r$ such that $JI^r = I^{r+1}$, and the minimal $r$ with this property is called the reduction number.
The property of being a reduction is tested by ${\tt isReduction\ }I$, and the reduction number, is then computed by ${\tt reductionNumber\ } I$. 

Now suppose that $\gm$ is a maximal ideal containing $I$. The special fiber ring is by definition
$\RR(I)/\gm\RR(I)$. It is a standard graded algebra ove the field  $k := R/\gm$, a quotient of 
$\Sym_R(F)/\gm = k[t_1,\dots,t_n]$ where, as before, $F$ is a free module of rank $n$ with a surjection to $M$. The defining ideal of the special fiber ring, and the ring itself, are computed  by the calls
${\tt specialFiberIdeal\ }I$ and ${\tt specialFiberRing\ }I$. 

The dimension of the special fiber ring is called the analytic spread of $I$, usually
denoted
$$
\ell(I) = {\tt analyticSpread\ }I.
$$
Northcott and Rees~\cite{} proved that if $k$ is infinite then there always exist reductions
generated by $\ell(I)$ elements, and this is the minimum possible number; these are called
minimal reductions. The smallest possible reduction number for $I$ with respect to an minimal reduction is  by definition ${\tt reduction number} I$ (this is always achieved by any ideal generated by $\ell(I)$ sufficiently general scalar linear combinations of the generators of $I$; but note that when $I$ is homogeneous but has generators of different degrees such linear combinations are sometimes necessarily inhomogeneous.)

An interesting special case occurs when $R$ is a graded ring over $k = R_0$ and the generators $g_1,\dots, g_n$ of $I$ are all homogeneous of the same degree. In this case the special fiber ring is easily seen to be equal to the subring $k[g_1,\dots,g_n]$ (usually \emph{not} a polynomial ring) generated by the elements $g_i$.

\def\G{{\mathbb G}}
For example, if $I$ is the ideal of $p\times p$ minors of a $p\times (p+q)$ matrix, then
the special fiber ring is equal to the homogeneous coordinate ring $\G$ of the Grassmannian of
$p$-planes in $p+q$ space. It follows that $\ell(I) = \dim \G = pq+1$, and it is known (reference??) that the reduction numberr of $I$ is ****.

\section{Distinguished subvarieties: {\tt distinguished, distinguishedAndMult}}
 \cite{F}

\section{A pathological example}       
     Here is an example from~\cite{EHU} of the pathological case of
     inclusions $\phi: M \to G$ and $\psi: M \to G'$ into free modules
     where $\ker(\phi) \neq \ker(\psi)$.
     In the following, any finite characteristic would work as well.p
        
     Here is an example from \cite{EHU}. where the symmetric algebra functor applied to different embeddngs of a module $M$ into free modules have different kernels, 
 By Theorem***, this is only possible over a field of positive characteristic. In the following, any positive characteristic would work as well.
\begin{verbatim}
     p = 5
     R = ZZ/p[x,y,z]/(ideal(x^p,y^p)+(ideal(x,y,z))^(p+1))
     M = module ideal(z) 
 \end{verbatim}
     It is easy to check that $M \cong R^1/(x,y,z)^p$.
     We write iota: M\to R^1 for the embedding as an ideal
     and psi for the embedding M \to R^2 sending z to (x,y).
    Example
     iota = map(R^1,M,matrix{{z}}) 
     psi = map(R^2,M,matrix{{x},{y}})
     ker iota
     ker psi
    Text
     Finally, the universal embedding is M \to R^3,
     sending z to (x,y,z):
    Example
     phi = universalEmbedding(M)
    Text
     We now compute the kernels of the three maps
     on symmetric algebras:
    Example
     Iiota = symmetricKernel iota;
     Ipsi = symmetricKernel psi;
     Iphi = symmetricKernel phi;
    Text 
     and check that the ones corresponding to phi and iota
     are equal, whereas the ones corresponding to psi and phi
     are not:
    Example
     Iiota == Iphi    
     Ipsi == Iphi
    Text
     In fact, they differ in degree p:

\section{Rees Algebras and Desingularization}

A map $B\to X$ of schemes is projective if it is the composition of a closed embedding $B\subset X\times \PP^n$ with the projection to $X$.
It is birational if it is generically an isomorphism. The inclusion of a ring into the Rees algebra
of an ideal corresponds to a map from Proj of the Rees algebra to spec of the ring, called a blowup, that is such a proper birational transformation, and in fact every proper birational transformation to an affine variety (or more generally to any scheme, if one works with sheaves of ideals) can be realized in this way.

The Theorem of embedded resolution of singularities, proven by Hironaka in characteristic 0 and conjectured in general, says that give any subvariety $X$ of a smooth variety $Y$, there is
a finite sequence of blowups 
$$
B_n \to \cdots B_2 \to B_1 \to Y
$$
of smooth subvarieties  and a component of the preimage of
$X$ in $B_n$ that is smooth. In the case of plane curves, this can be done with a sequence of blowups of closed points. But in fact *any* sequence of blowups can be replace with a single blowup (\cite[Theorem***]{Hartshorne} of a more complicated ideal. We illustrate with the desingularization of a tacnode (the union of two smooth curves that meet with a simple tangency.)

\begin{example}
Blowing-up $(x^2,y)$ in k[x,y] desingularlizes the tacnode $x^2-y^4$ in a single step. 
\end{example}
\begin{verbatim}
     R = ZZ/32003[x,y]
     tacnode = ideal(x^2-y^4)
     mm = ideal(x^2,y)
     B = reesAlgebra mm
     irrelB = ideal vars B
     totalTransform = substitute(tacnode, B)
     netList (D = decompose totalTransform)
     properTransform = saturate(totalTransform, promote(mm,B))
     allvarsB = promote(vars R,B)|vars B
     jacobianMatrix = diff(allvarsB, transpose gens properTransform)
     codim properTransform
     sing = saturate(minors(1,jacobianMatrix)+properTransform, irrelB)
     assert(sing = ideal 1_(ring sing))
  \end{verbatim}
The last line asserts that the singular locus of the the variety ``properTransform'' is empty;
that is, the the scheme defined by "properTransform" is smooth (in this case it is the union
of two disjoint smooth curves.)

\let\thefootnote\relax\footnote{
\noindent A\S Subject Classification:\\
Primary: 13C40, 13H10, 14M06, 14M10;
Secondary: 13D02 , 13N05, 14B12, 14M12.\smallbreak
The author is grateful to the
National Science Foundation for partial support.<}
\bibliographystyle{alpha}
\begin{thebibliography}{ABC99}
%\bibitem{Bass1963} Bass, Hyman
%On the ubiquity of Gorenstein rings.
%Math. Z. 82 1963 8\UTF{2013}28.


%\bibitem[AH]{AH} L.~Avramov and J.~Herzog, The Koszul algebra of a codimension 2 embedding. Math. Z. 175 (1980) 249--260.
%
%%\bibitem[B]{B} R.~Berger **** Crelle.
%
%\bibitem[AK]{AK} A. Altman and S. Kleiman, Introduction to Grothendieck duality, Springer Lect. Notes in Math. 146 (1970).
%
%\bibitem[AN]{AN} M. Artin and M. Nagata, Residual intersections in Cohen-Macaulay rings.
%J. Math. Kyoto Univ. 12 (1972) 307--323.
%
%\bibitem[B]{B} R.~Berger,  \"Uber verschiedene Differentenbegriffe. S.-B. Heidelberger Akad. Wiss. Math.-Nat. Kl. (1960/61) 1--44.
%
%\bibitem[BE1]{BE1} D. Buchsbaum and D.~Eisenbud, Generic free resolutions and a family of generically perfect ideals.
%Advances in Math. 18 (1975) 245--301.
%
%\bibitem[BE2]{BE2} D. Buchsbaum and D.~Eisenbud,
%What annihilates a module?
%J. Algebra 47 (1977) 231--243.
%
%\bibitem[B]{B} R.-O. Buchweitz, Contributions \`{a} la th\'{e}orie des singularit\'{e}s, th\`{e}se d' \'{E}tat, Universit\'{e} Paris VII, 1981.
%
%\bibitem[C]{C} M. Chasles, Construction des coniques qui satisfont \`a cinque conditions, C. R. Acad. Sci. Paris 58 (1864) 297--308.

\bibitem[CEU]{CEU} -- our paper with Chardin

%\bibitem[CNT]{CNT} M.~Chardin, J.~Na�liton, and Q.~H.~Tran, Cohen-Macaulayness and canonical module of residual intersections,
%arXiv:1701.08087.
%	
%\bibitem[CP]{CP} A.~Corso and C.~Polini,
%On residually $S_{2}$ ideals and projective dimension one modules.
%Proc. Amer. Math. Soc. 129 (2001)1309--1315.
%
%\bibitem[E]{E} D.~Eisenbud, Commutative Algebra with a View Toward Algebraic Geometry, GTM 195,
%Springer-Verlag 1994.
%
%\bibitem[EE]{EE} D.~Eisenbud, E.~G~Evans, Jr., Generating modules efficiently: Theorems from algebraic K-theory. J. Alg.. 27 (1973) 278--305.
%
%\bibitem[EHo]{EHo} D.~Eisenbud and M.~Hochster, A Nullstellensatz with nilpotents and Zariski's main lemma on holomorphic functions,
%J. Algebra 58 (1979) 157--161.
%

 \bibitem[EHU]{EHU} D.~Eisenbud, C.~Huneke and B.~Ulrich, What is the
     Rees algebra of a module? Proc. Am. Math. Soc. 131, 701--708, 2002.
  
\bibitem[EU]{EU} D.~Eisenbud and B.~Ulrich, Duality and socle generators for residual intersections. To appear.

\bibitem[F]{F} W.~Fulton, Intersection Theory ****.

%\bibitem[F]{F} H.~Flenner,
%Die S\"atze von Bertini f\"ur lokale Ringe.
%Math. Ann. 229 (1977) 97--111.
%
%%\bibitem[FHJ]{FHJ} L.~Fouli, C.~Huneke and M.~Johnson, In Preparation.
%
%\bibitem[H]{H} S. H. Hassanzadeh, Cohen-Macaulay residual intersections and their Castelnuovo-Mumford
% regularity, Trans. Amer. Math. Soc.  364  (2012) 6371--6394.
%
% \bibitem[HN]{HN} S. H. Hassanzadeh and J. Na\'eliton,
%Residual intersections and the Annihilator of Koszul Homologies, Preprint,  arXiv:1405.4586.
%
% \bibitem[HSV1]{HSV1} J.~Herzog, A.~Simis and W.~Vasconcelos, Approximation complexes of blowing-up rings,
% J. Alg. 74 (1982) 466--493.
%
%\bibitem[HSV2]{HSV2} J.~Herzog, A.~Simis and W.~Vasconcelos, Approximation complexes of blowing-up rings II,
% J. Alg. 82 (1983) 53--83.
%
%\bibitem[HVV]{HVV} J.~Herzog, R.~Villarreal and W.~Vasconcelos, Ideals with sliding depth,
%Nagoya Math. J. 99 (1985), 159--172.


% \bibitem[H1]{H1}C.~Huneke,
%On the associated graded ring of an ideal.
%Illinois J. Math. 26 (1982) 121--137.


% \bibitem[H1]{H1} C.~Huneke, Linkage and the Koszul homology of ideals, Amer. J. Math. 104 (1982) 1043--1062.
%
% \bibitem[H2]{H2} C.~Huneke, Strongly Cohen-Macaulay schemes and residual intersections,
%Trans. Amer. Math. Soc. 277 (1983) 739--763.
%
% \bibitem[HU]{HU} C.~Huneke and B.~Ulrich,  Residual intersections, J. reine angew. Math. 390 (1988) 1--20.

%\bibitem[Jo]{Jo} M.~Johnson, 

%\bibitem[K1]{K1} E.~Kunz, Holomorphe Differentialformen auf algebraischen Variet\"aten mit Singularit\"aten I,
%Manuscripta Math. (1975) 91--108.
%Thm 4.1 is a special case of the invariance of the different on change of Noether normalization.

%\bibitem[K1]{K1} E.~Kunz, K\"ahler Differentials, Vieweg Advanced Lectures, Braunschweig,1986.

%\bibitem[K3]{K3} E.~Kunz, Differentialformen auf algebraischen Variet\"aten mit Singularit\"aten,
%Abh. Math. Sem. Univ. Hamburg 47 (1978), 42--70.
%Satz 3.1 is the more general form.

%\bibitem[K2]{K2} E.~Kunz, Residues and duality for projective algebraic varieties, University Lect. Series 47 (2009).

% \bibitem[KU1]{KU1} A.~Kustin and B.~Ulrich, If the socle fits, J. Alg. 147 (1992) 63--80.

\bibitem[KU]{KU} A.~Kustin and B.~Ulrich, A family of complexes associated to an almost alternating map, with applications to residual intersections, Mem. Amer. Math. Soc. 95 (1992).

%\bibitem[LS]{LS} J.~Lipman and A.~Sathaye,  Jacobian ideals and a theorem of Brian�on-Skoda. Michigan Math. J.  28  (1981) 199--222.
%
%\bibitem[M2]{M2}
%Macaulay2{\thinspace}--{\thinspace}a system for computation in
%  algebraic geometry and commutative algebra programmed by D.~Grayson and M.~Stillman,
%\href{http://www.math.uiuc.edu/Macaulay2/}{\tt  http://www.math.uiuc.edu/Macaulay2/}
%
%\bibitem [No]{No} E.~Noether, 
%Idealdifferentiation und Differente. 
%J. Reine Angew. Math. 188, (1950) 1--21. 
%
% \bibitem[PS]{PS} C.~Peskine and L.~Szpiro, Liaison des vari\'et\'e alg\'ebriques,  Invent. Math. (1974) 271--302.
%

\bibitem[Rees]{Rees} On a problem of Zariski, Illinois J. Math. (1958) 145-149).
 
%
%\bibitem[SS0]{SS0} G.~Scheja and U.~Storch, Lokale Verzweigungstheorie. 
%Vorlesungen \"uber Kommutative Algebra (Wintersemester 1973/74). Schriftenreihe des Mathematischen Institutes der Universit\"at Freiburg, No. 5. Institut des Math\'ematiques, Universit\'e de Fribourg, Fribourg, 1974.

%\bibitem[SS]{SS} G.~Scheja and U.~Storch, \"Uber Spurfunktionen bei vollst\"andigen Durchschnitten,
%J. Reine Angew. Math. 278 (1975) 174--190.
%
%
%\bibitem[S]{S} E.~Sernesi, The local cohomology of the jacobian ring,
%Doc. Math. 19 (2014) 541--565.
%
%\bibitem[SW]{SW} D.~van Straten and T.~Warmt, Gorenstein-duality for one-dimensional almost complete intersections---with an application to non-isolated real singularities,
%Math. Proc. Cambridge Philos. Soc. 158 (2015) 249--268.
%
%\bibitem[T]{Tate} J.~Tate, Appendix to B.~Mazur and L.~Roberts,
%Local Euler characteristics,
%Invent. Math. 9 (1969/1970) 201--234.

\bibitem[U]{U} B.~Ulrich, Artin-Nagata properties and reductions of ideals, Contemp. Math. 159 (1994) 373--400.

\bibitem[VV]{VV} P.~Valabrega and G.~Valla, Form rings and regular sequences, Nagoya Math. J. 72 (1978) 91--101.

%\bibitem[Wa]{Wa} J.~Watanabe, A note on Gorenstein rings of embedding codimension three, Nagoya Math.
%J. 50 (1973) 227--232.
%
%\bibitem[W]{W} T.~Warmt,
%Gorenstein-Dualit\"at und topologische Invarianten von Singularit\"aten, Dissertation, Johannes Gutenberg Universit\"at Mainz, Mainz, 2002. Cuvillier Verlag, G\"ottingen, 2003.
%
%
%\bibitem[Wi]{Wi} H.~Wiebe, \"{U}ber homologische Invarianten
%lokaler Ringe, Math. Ann. 179 (1969), 257--274.


\end{thebibliography}

\bigskip

\vbox{\noindent Author Addresses:\par
\smallskip
\noindent{David Eisenbud}\par
\noindent{Mathematical Sciences Research Institute,
Berkeley, CA 94720, USA}\par
\noindent{de@msri.org}\par
}

\end{document}



\begin{theorem}\label{Dedekind complementary module2}
In addition to the assumptions in the definition of the Dedekind different above, suppose that $A$ is a regular local ring and $R$ is finite
as an $A$-module.
\begin{enumerate}
\item If $R$ is Gorenstein, then $\fD_D(R/A) = \fD_N(R/A)$.
\item If $R$ is locally a complete intersection,
 then $\fD_N(R/A) = \fD_K(R/A)$ is generated by the image $\Delta$ in $R$
 of a Jacobian determinant, and $\fC(R/A) = R \, \Delta ^{-1}$ .
 \end{enumerate}
\end{theorem}
%\begin{theorem}\label{Dedekind complementary module2}
% Under the assumptions in the definition of the Dedekind different, above,
%\begin{enumerate}
% \item If $R$ is a relative complete intersection over $A$,
% then $\fD_K(R/A) = \fD_N(R/A)$.
% \item If $R$ is Gorenstein, then $\fD_N(R/A) = \fD_D(R/A)$.
%\end{enumerate}
%Thus, if $R$ is a complete intersection, then $\fC(R/A) = 1/J$, where $J$ is the Jacobian
%determinant.
 \end{theorem}

%\david{I think we need more assumptions, along the line of $A$ being a power series ring over a field of char 0: otherwise $R$ being a complete
%intersection is not the same as$R = A[\seq y n]/(F)$ where $F = F_1,\dots, F_{n}$ is a regular sequence in $A[\seq y n]$.
%
%I added that last statement of the Theorem, which seems an immediate consquence of the two items, and thus I did not add any explicit proof. OK?}

\begin{proof} 1) Since $R$ is a Cohen-Macaulay ring, it is a free $A$-module, and since $R$ Gorenstein and semilocal, it follows that ${\rm Hom}_A(R,A)
\cong R$ as an $R$-module. Thus we may apply Lemma~\ref{Noether}. As
${\rm Hom}_A(R,A)=\fC(R/A)\; \Trace_{R/A}$ by the definition of the
complementary module, the lemma shows that
$\fD_N(R/A) \, \fC(R/A) =R$, which gives $\fD_N(R/A)=\fC(R/A)^{-1}=\fD_D(R/A)$.

\smallskip
\noindent
2) Since $R$ is semilocal, locally a complete intersection, and a finite $A$-module, we have $R \cong A[\seq y n]/(F)$ for some regular sequence $F = F_1,\dots, F_{n}$ of length $n$ in the polynomial ring $A[\seq y n]$. Write
$\Delta$ for the image in $R$ of the Jacobian determinant of
$F_1,\dots, F_{n}$ with respect to $\seq y n$.
Let $\bD$ be the kernel of the multiplication map $\mu: R^e = R\otimes_AR \to R$ as above.
The preimage $\tilde \bD$ of $\bD$ in
$A[\seq y n]\otimes_{A} R$ is the kernel of the natural map to $R$,
so it is also generated by a regular sequence of length $n$, say
$\tilde \bD = (G_1,\dots, G_n)$
and $\tilde \bD$ contains the regular sequence $F\otimes 1$.

The preimage in
$A[\seq y n]\otimes_{A} R$ of the
annihilator of $\bD$  may thus be written $(F\otimes 1) : \tilde \bD$. This ideal quotient is generated by $F\otimes 1$ and the determinant of
any matrix $\Theta$ expressing the elements of $F\otimes 1$ as  linear combinations of the elements of $G$ \david{give reference? Peskine-Szpiro?}. Since
$$
\bD\otimes_{R^e}R = \bD/\bD^2 \cong \Omega_{R/A},
$$
the image in $R$ of  $\Theta$ is a presentation matrix
of $\Omega_{R/A}$, and thus (up to a unit) has determinant $\Delta$.
Hence we have shown that $\fD_N(R/A)= R \, \Delta =\fD_K(R/A)$.

Using part (1) and the fact that the fractional ideal $\fC(R/A)$ is
principal, hence invertible, it also follows that $\fC(R/A)= R \, \Delta^{-1}$.
\end{proof}
